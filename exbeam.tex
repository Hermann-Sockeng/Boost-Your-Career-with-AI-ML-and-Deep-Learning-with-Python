\documentclass[9pt]{beamer}

\usetheme{uha}

\usepackage[french,english]{babel}
\usepackage[T1]{fontenc}
\usepackage[utf8]{inputenc}
\usepackage{amsmath}
\usepackage{amsfonts}
\usepackage{amssymb}
\usepackage[loadonly]{enumitem}

\newlist{perso}{enumerate}{1}
\setlist[perso]{label=\arabic*--}

\title{Titre de la présentation}
\subtitle{Sous-titre}
\date{Date d'aujourd'hui}
\author{Auteur de la présentation}
\institute{Université de Haute Alsace}

\begin{document}

% Title page 
\begin{frame}[plain, noframenumbering]
	\titlepage
\end{frame}

\section{Introduction}

\begin{frame}[fragile]{beamer UHA}
	
	Le thème beamer uha s'inspire du thème Metropolis qui lui même s'inspire du thème HSRM de Benjamin Weiss.

	Le thème se charge de la manière suivante :

	\begin{verbatim}
	\documentclass[9pt]{beamer}
	\usetheme{uha}
	\end{verbatim}
\end{frame}

\begin{frame}[fragile]{Section}
	Chaque section permet de regrouper des diapositives d'un même sujet.
	\begin{verbatim}
	\section{Les éléments}
	\end{verbatim}
\end{frame}

\section{Les éléments}

\begin{frame}[fragile]{Typographie}
	\begin{verbatim}
	Le thème uha permet de mettre en \emph{valeur du texte}, 
	de mettre en \textbf{gras du texte} et même 
	\alert{d'alerter le lecteur}.
	\end{verbatim}
	Le thème uha permet de mettre en \emph{valeur du texte}, de mettre en \textbf{gras du texte} et même \alert{d'alerter le lecteur}.
\end{frame}

\begin{frame}{Les listes classiques}
	\begin{minipage}[t]{0.45\linewidth}
		\alert{Une liste}
		\begin{itemize}
				\item Premier,
				\item Second,
				\item Troisième.
		\end{itemize}
	\end{minipage}
	\hfill
	\begin{minipage}[t]{0.45\linewidth}
		\alert{Une énumération}
		\begin{enumerate}
				\item Premier
				\item Second
				\item Troisième
		\end{enumerate}
	\end{minipage}
	
	\vspace{2em}

	\alert{Une description}
	\begin{description}
			\item [UHA] Université de Haute Alsace
			\item [EURCOR] Le campus européen
	\end{description}
\end{frame}


\begin{frame}[fragile]
	\frametitle{Personnalisation des listes avec enumitem}
	Le paquet \alert{enumitem} peut être utilisé si et seulement l'option \alert{loadonly} est activé, comme indiqué ci-dessous.
\begin{verbatim}
\usepackage[loadonly]{enumitem}
\end{verbatim}
	Etant donné que le paquet est chargé avec l'option \alert{loadonly}, la définition de nouvelles liste devient possible.
	\begin{perso}
			\item Le premier,
			\item le second.
	\end{perso}
\end{frame}

\begin{frame}{Les blocks}
	\begin{exampleblock}{Exemple}
		Un exemple
	\end{exampleblock}
	\begin{alertblock}{Exemple}
		Une alerte
	\end{alertblock}
	\begin{block}{Exemple}
		Un block
	\end{block}
\end{frame}
\end{document}

%%% Local Variables:
%%% mode: latex
%%% TeX-master: t
%%% End:
